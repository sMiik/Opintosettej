% Finnish or English, gives the right language to title page and Finnish enables special chars ä ö å.
\documentclass[a4paper,finnish,12pt]{article}
\usepackage[T1]{fontenc}
\usepackage[utf8]{inputenc}
\usepackage{babel}

\usepackage{mathtools}
\usepackage[lstlisting]{/home/micsu/Dropbox/Opinnot/gittisetit/LaTeX/mcode}
\usepackage{amsfonts,amssymb,amsbsy}
\usepackage{fancyhdr}
\usepackage[a4paper]{geometry}
\usepackage[RGB,ELEC]{/home/micsu/Dropbox/Opinnot/gittisetit/LaTeX/aaltologo}

\begin{document}

\thispagestyle{empty}

\begin{titlepage}
    \centering
    \vspace*{11\baselineskip}
    \huge
    \bfseries
    Harjoitteluraportti \\
    \Large
    \vfill
    Appelsiini Finland Oy \\
    \vfill
    \small
    Automaatio- ja systeemitekniikka \\
    \normalfont
    \vfill
    Miikka Eloranta \\
    80294A \\[2\baselineskip]
    \textbf{\today} \\[2\baselineskip]
    \vfill
	\AaltoLogoSmall{1}{?}{aaltoPurple}


\end{titlepage}

\pagebreak

Sain osa-aikaisen työpaikan Appelsiini Finland Oy:stä jo ensimmäisen opiskeluvuoden keväänä –maaliskuussa 2009. Siitä lähtien olen työskennellyt Appelsiinilla osa-aikaisena opintojen ohella, sekä kesäisin ja muina opiskelun lomajaksoina täyspäiväisesti.  Sain neuvoteltua opintojen ohelle sopivan työsopimuksen, jossa työtuntimääräni on määritetty 0-40 tuntiin viikossa, koska en opintojen vuoksi kyennyt tarkkoja tuntimääriä arvioimaan periodeittain vaihtuvan opiskelukalenterini vuoksi. Alussa työtä tuli tehtyä vain parina iltana viikossa, mutta vastuun kasvettua myös työmäärät lisääntyivät ja vuonna 2013 työtuntien keskiarvo koko vuodelta oli lähes 30 tuntia viikossa.

Appelsiini Finland Oy on keskisuuriin yrityksiin keskittynyt, vuonna 1999 perustettu IT-palvelutalo, jossa tarjotaan ulkoistuksia laajasti koko IT-palvelualueelta; laitteita ja lisenssejä, sovellushallintaa ja –kehitystä, sekä tietohallinnon konsultointia ja työn tuottavuuden kehittämiseen liittyviä palveluita. Vuonna 2010 Tietoviikko valitsi Appelsiinin vuoden TiVi-yritykseksi, ja vielä samana vuonna se ostettiin yksityisiltä osakkeenomistajilta Elisa Oyj:n tytäryhtiöksi. Appelsiini on kasvanut koko olemassaolonsa ajan noin 30\% vuositahdilla henkilöstössä aina neljän perustajan pienestä yrityksestä nykyiseksi yli 300 henkilöä työllistäväksi Elisa Oyj:n IT-liiketoimintayksiköksi. Aiemmin Elisan IT-liiketoiminnasta vastannut Elisa Links Oy fuusioitui osaksi Appelsiinia vuoden 2013 alussa.

Appelsiinin työurani alussa kuuluin osaksi Tarjooma-tiimiä, joka vastasi tuolloin tuotekehityksestä ja -markkinoinnista. Käytännössä toimin silloisen esimieheni sanoin ”Tarjooman yhden hengen sovelluskehitysyksikkönä” ja kehitin itsenäisesti ohjelmistoa myynnin tueksi suunnitellen ja kehittäen itsenäisesti tietokantarakenteen, ohjelmistologiikan ja käyttöliittymän.

Hiljalleen Appelsiinin työurani aikana sain osakseni lisää vastuuta ja kehitettäviä ja ylläpidettäviä järjestelmiä tuli lisää. Syksyllä 2012 keskustelimme silloisen esimieheni ja sisäisistä järjestelmistä vastaavan tiimin esimiehen kanssa ja tulimme siihen lopputulokseen, että koska varsinaisesti omat työkuvani eivät liittynyt Tarjooman tekemisiin, olisi siirtyminen sisäisistä järjestelmistä vastaavaan tiimiin hyvä päätös. Samoihin aikoihin siirtymiseni kanssa organisaatiossa tapahtui rakennemuutoksia ja sisäiset järjestelmät muutettiin tietohallinnoksi, joka oli aiemmin kattanut yhden henkilön, sekä muutaman oman työn ohella tietohallinnon töitä tekevän henkilön.

Koska organisaatio oli kasvanut kovaa vauhtia ja toimintamallit olivat pääasiassa edelleen pienen yrityksen mallilla, tietohallinnossa piti ryhtyä muuttamaan koko organisaation toimintatapoja tehokkaammiksi. Ensi alkuun määräsimme kaikki työ- ja tukipyynnöt jätettäviksi toiminnanohjausjärjestelmään, jotta niistä olisi historiatietoa ja kunkin yksilön työmääriä pystyttiin seuraamaan. Suurimmaksi osaksi jokaisella tietohallinnon jäsenellä oli itsenäisiä projekteja ja järjestelmiä vastuullaan, eikä viikottainen palaveri ollut riittävä muiden työmäärien selvittämiseksi. Yksikön vastuulle siirrettiin kaikki laitetilaukset ja yrityksen sisäinen käyttäjätuki.

Myöhemmin rupesimme selvittämään kaikkia organisaation sisäisiä järjestelmiä, sekä pohtimaan niiden tärkeyttä ja roolia organisaation kannalta. Pidimme näiden osalta tiimin kehityspäivät, jossa päätimme järjestelmistä vastaaville henkilöille roolit, määritimme kunkin roolin vastuualueet ja järjestimme kaikki järjestelmät asteikolle sen mukaan, kuinka kriittisiä ne koko yrityksen kannalta ovat. Lisäksi kävimme läpi kaikki järjestelmiin liittyvät henkilöt, roolitimme nämä järjestelmäkohtaisesti ja lopulta tietohallintopäällikkö esitteli ja hyväksytti ehdotuksemme johtoryhmässä.
Viime aikoina olen itse keskittynyt jo kesällä 2011 alkaneeseen sovelluskehitysprojektiin, jossa kehitetään järjestelmää yrityksen henkilökunnan osaamisten hallintaan ja raportointiin, kehittymisen seurantaan ja liiketoiminnan tukemiseen. Järjestelmän kehitys oli alkuvaiheessa erittäin kankeaa, sillä määrittelyt oli tehty vain ylätasolla, eikä tarkempaa tietoa konkreettisistä näkymistä tai halutuista toiminnoista saanut kuin silloisen vastuuhenkilön hihasta nykimällä. Vasta vuoden 2012 lopulla saimme organisoitua järjestelmän kehitystä perustamalla sen kehityksestä ja määrityksistä vastaavan ohjausryhmän. Tämän lisäksi vastuulleni on tullut erinäisiä pienempiä projekteja sisältäen muun muassa kampanja- ja www-sivujen kehitystä ja ylläpitoa,  sisällönhallintajärjestelmien kehitystä ja integraatioita. Tämän lisäksi työkuvaani kuuluu muita tietohallinnon töitä, sisältäen muun muassa käyttöoikeuksien muokkausta ja järjestelmäylläpitoa ja –kehitystä.

Työskennellessäni Appelsiinilla olen saanut selkeämmän kuvan määriteltyjen prosessien tärkeydestä kaikessa kehityksessä ja toiminnassa \textendash sekä nähnyt mihin niiden puutteellisuus voi johtaa. Myös töiden organisoitu suunnittelu, työskentelyn ohjaus ja dokumentointi on tärkeää, jotta vältytään ylimääräiseltä selvittelyltä myöhemmin jatkokehitystä suunnitellessa. 

\end{document}
