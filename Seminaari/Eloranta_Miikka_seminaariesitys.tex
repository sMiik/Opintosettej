\documentclass[first=purple,second=dblue,logo=redquo]{aaltoslides}

\usepackage[finnish]{babel}
\usepackage[utf8]{inputenc}
\usepackage[T1]{fontenc}
\usepackage{url}
\usepackage{lastpage}
\usepackage{hyperref}
\usepackage[style=finnish]{csquotes}
\usepackage{tikz}
\usetikzlibrary{positioning}
\usetikzlibrary{calc}
\usetikzlibrary{arrows}
\usetikzlibrary{decorations.pathmorphing,decorations.markings}
\usetikzlibrary{shapes}
\usetikzlibrary{patterns}

\usepackage[backend=biber,style=oscola]{biblatex}

\makeatletter
\renewbibmacro*{footcite}{%
  \bbx@resetpostnotedelim
  \usebibmacro{cite:citepages}%
  \global\togglefalse{cbx:loccit}%
  \ifboolexpr {test {\ifciteseen} or test {\ifciteibid}}
    {\ifboolexpr {test {\iffieldundef{shorthand}} 
                  or test {\bbx@ifnottrackingcites}}
       {\ifthenelse{\ifciteibid\AND\NOT\iffirstonpage}
          {\usebibmacro{footcite:ibid}\bbx@unsetpostnotedelim}
          {\usebibmacro{footcite:note}}}%
       {\usebibmacro{footcite:shorthand}}}
    {\usebibmacro{footcite:full}%
     \usebibmacro{footcite:save}}}

\renewbibmacro*{postnote}{%
  \ifboolexpr {test {\iffieldundef{postnote}} or
               test {\iftoggle{cbx@postnoteprinted}}}%
    {}
    {\ifboolexpr{test {\ifnumequal{\value{bbx@suppresspostnotedelim}}{1}}}
       {\setunit{\addspace}%
        \bbx@resetpostnotedelim}
       {\setunit{\postnotedelim}}%
     \usebibmacro{postnotepagination}}%
  \global\toggletrue{cbx@postnoteprinted}}
\makeatother



\addbibresource{presisrefet.bib}

\title{Osaamisen kehityksen hallinta}

\author[M. Eloranta]{Miikka Eloranta}
\institute[AS]{Automaatio- ja systeemitekniikka\\
Sähkötekniikan korkeakoulu, Aalto-yliopisto}

\aaltofootertext{Osaamisen kehityksen hallinta}{\today}{\arabic{page}/\pageref{LastPage}\ }

\date{20.03.2015}

\begin{document}

\aaltotitleframe

\begin{frame}{Sisältö}
\begin{itemize}
\item Yleistä
\begin{itemize}
\item Sisällön kuvaus
\item Työn sisällön painopisteet ja taustat
\end{itemize}
\item Osaamiset
\begin{itemize}
\item Osaamisten ja niiden laadun määritys
\item Osaamisten jaottelu
\item T-malli
\end{itemize}
\item Kehittyminen
\begin{itemize}
\item Henkilöstön kehityssuunnitelmat
\item Yrityslaajuiset tavoitteet
\item Urakehitysmahdollisuudet
\end{itemize}
\item Keskitetyn osaamisten hallinnan hyödyt ja mahdollisuudet
\end{itemize}
\end{frame}

\begin{frame}{Yleistä}
\begin{itemize}
\item Tässä työssä keskitytään ICT\footnote{\tiny{Tieto- ja viestintäteknologia (Information and Communications Technology)}}-palveluyrityksiin ja esitetyt osaamisjaottelut sekä tavoite- ja kehittymisprosessit on määritelty niiden mukaisesti.
\begin{itemize}
\item Osittain yleisemmän tason asiaa, joka soveltuu muillekin aloille.
\end{itemize}
\item Työssä esitellyt menetelmät, mittaroinnit ja kuvaukset on suunniteltu osaamisten kehityksen hallintaa varten kehitettyä järjestelmää varten, mutta työssä esitellään vain näitä -- ei itse järjestelmää.
\end{itemize}
\end{frame}

\begin{frame}{Osaamiset}
\begin{itemize}
\item Parhaat osoitukset osaamisesta ICT-maailmassa ovat sertifioitumiset ja referenssit
\begin{itemize}
\item Tarvitaan muitakin kuin teknologiaosaamisia \\ $\rightarrow$ Kaikkiin osaamisiin ei voi sertifioitua
\item Sertifioituminen ei ole varma tae käytännön osaamisesta
\end{itemize}
\item Palveluyrityksen valinnassa asiakas huomioi myös kumppanit, kumppanuustasot ja olemassaolevat asiakkuudet -- ennen kaikkea onko yrityksellä asiakkaana saman toimialan yrityksiä \footcite{ICT-haasteet}
\item Saman koulutustaustan ihmiset voivat erota huomattavasti toisistaan \footcite{self-assessment_in_skill_measurement}
\end{itemize}
\end{frame}

%% Lähdeluettelo
\begin{frame}{Viitteet}
\printbibliography
\end{frame}
\end{document}
