\documentclass[first=purple,second=dblue,logo=redquo]{aaltoslides}

\usepackage[finnish]{babel}
\usepackage[utf8]{inputenc}
\usepackage[T1]{fontenc}
\usepackage{url}
\usepackage{lastpage}
\usepackage{hyperref}
\usepackage[style=finnish]{csquotes}
\usepackage{tikz}
\usetikzlibrary{positioning}
\usetikzlibrary{calc}
\usetikzlibrary{arrows}
\usetikzlibrary{decorations.pathmorphing,decorations.markings}
\usetikzlibrary{shapes}
\usetikzlibrary{patterns}

\title{Osaamisen kehityksen hallinta}

\author[M. Eloranta]{Miikka Eloranta}
\institute[AS]{Automaatio- ja systeemitekniikka\\
Sähkötekniikan korkeakoulu, Aalto-yliopisto}

\aaltofootertext{Osaamisen kehityksen hallinta}{\today}{\arabic{page}/\pageref{LastPage}\ }

\date{20.03.2015}

\begin{document}

\aaltotitleframe

\begin{frame}{Sisältö}
\begin{itemize}
\item Yleistä
\begin{itemize}
\item Työn sisällön painopisteet ja taustat
\end{itemize}
\item Osaamiset
\begin{itemize}
\item Osaamisten ja niiden laadun määritys
\item Osaamisten jaottelu ja visualisointi
\end{itemize}
\item Kehittyminen
\begin{itemize}
\item Henkilöstön kehityssuunnitelmat
\item Yrityslaajuiset tavoitteet
\item Urakehitysmahdollisuudet
\end{itemize}
\item Keskitetyn osaamisten hallinnan hyödyt ja mahdollisuudet
\end{itemize}
\end{frame}

\begin{frame}{Yleistä}
Tässä työssä keskitytään ICT-palveluyrityksiin ja esitetyt osaamisjaottelut sekä tavoite- ja kehittymisprosessit on määritelty niiden mukaisesti.
\begin{itemize}
\item Osittain yleisemmän tason asiaa, joka soveltuu muillekin aloille.
\end{itemize}

\end{frame}

\end{document}