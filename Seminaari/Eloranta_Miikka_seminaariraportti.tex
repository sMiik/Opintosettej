% Finnish or English, gives the right language to title page and Finnish enables special chars ä ö å.
\documentclass[a4paper,finnish,12pt]{article}
\bibliographystyle{ieeetr}
\usepackage[T1]{fontenc}
\usepackage[utf8]{inputenc}
\usepackage[finnish]{babel}

\usepackage{mathtools}
\usepackage[lstlisting]{../LaTeX/mcode}
\usepackage{amsfonts,amssymb,amsbsy}
\usepackage{fancyhdr}
\usepackage[a4paper]{geometry}
\usepackage[RGB,ELEC]{../LaTeX/aaltologo}
\usepackage{graphicx}
\usepackage{caption}
\usepackage{subcaption}

\begin{document}

\thispagestyle{empty}

\begin{titlepage}
    \centering
    \vspace*{10\baselineskip}
    \Large
    \bfseries
    AS-0.3100 \\ Automaatio ja systeemitekniikan seminaari \\
    \vspace{\baselineskip}
    \huge
    Osaamisen kehittymisen hallinta \\
    [1.5\baselineskip]
    \normalfont
    \vfill
    \small
    Automaatio- ja systeemitekniikka \\
    \vfill
    Miikka Eloranta \\
    80294A \\[2\baselineskip]
    \textbf{\today} \\[2\baselineskip]
    \vfill
	\AaltoLogoSmall{1}{?}{aaltoPurple}


\end{titlepage}

% DOCUMENT START

\clearpage

\tableofcontents

\clearpage

\section{Johdanto}


Yritysmaailmassa oman työnsä tehostamiseksi on muiden työntekijöiden tuntemisen lisäksi hyvä tuntea myös kollegoidensa osaamisalueet. Pienikokoisissa yrityksissä, pienissä startupeissa tai yksittäisenä yrittäjänä toimiessa tämä on helppoa, koska ihmisiä on vähän tai yksittäisenä yrittäjänä oman osaamisalueen ulkopuoleiseen tekemiseen valitaan juuri siihen aihealueeseen erikoistunut kumppani. Pienyrityksessä kullakin on tarkka vastuualue ja jokainen on osaltaan tärkeä kokonaisuuden kannalta, joten osaamisalueiden tuntemus on oman työnkin kannalta ehdoton vaatimus. Suurissa ja etenkin nopeasti henkilöstömäärällisesti kasvavissa yrityksissä muiden työntekijöiden -- saati näiden kunkin osaamisalueen -- tuntemus on sen sijaan selvästi hankalampaa.

Kollegoiden osaamisten tuntemisen tärkeys korostuu ennen kaikkea asiakaspalveluyrityksissä. Kovassa kilpailutuksessa on osattava kertoa laajemmin koko yrityksen osaamistarjonnasta ja myös olemassa olevien asiakkaiden muuttuviin tarpeisiin on pystyttävä reagoimaan nopeasti -- joko etsimällä tarpeen täyttävä osaaja yrityksen sisältä, rekrytoimalla tai kehittämällä osaamista. Koska kuitenkaan kaikkien yksittäisten asiakkaiden tarpeiden muuttuessa ei osaamista voi välttämättä kehittää tarpeiden mukaan, on osaamisen kehittämistä osattava ennakoida ja valita tärkeimmän kehityssuunnat.

Sen lisäksi, että kollegoiden osaaminen on hyvä tuntea, tulee koko yrityksen osaamisten olla keskitetysti hallittua. Tällöin oman osaamisalueen ulkopuoleisissa toimissa tietädetään, keneltä pyytää apua ja asiakkaalle voidaan nopeasti ilmoittaa tietyn osaamisen hallitsevat henkilöt. Monesti asiakkaan kilpailuttaessa palveluyrityksiä tullaan varmistamaan, että tällä on riittävästi asiakkaan tarpeet täyttäviä osaajia. Lisäksi tämä on hyödyllistä yrityksen kehitystarpeita kartoittaessa.

Teknologiaosaamisten osalta korkean luokan asiantuntijuus ei saisi keskittyä laajalta osin liian pieneen osaan yrityksen henkilöstöä. Vaikka eri työrooleissa asiantuntijuuden syvyys vaihtelee laajalti, ei yksittäinen henkilö saa olla omassa roolissaan korvaamaton. Siinä missä asiakaspalvelija voi vastata laajalti eri teknologioihin tai asiakasympäristöihin liittyvistä perusongelmista, täytyy jollain olla osaamista myös tarkasti kunkin ympäristön arkkitehtuurista. Kuitenkin, jos yksi ja sama henkilö on suunnitellut ja rakentanut kaikkien asiakasympäristöjen tietyt osa-alueet, kostautuu tämä valtavana ongelmana esim. henkilön irtisanoutuessa.

Osaamiset on siis saatava hallitusti jaettua laajemmalle määrälle henkilöstöä pitäen kuitenkin mielessä kunkin työroolin kannalta olennaiset osaamisalueet ja eri osaamisten tasot. Jotta näitä saadaan hallitusti ja jatkuvasti seurattua ja kehitettyä, on kaikkea henkilöstön osaamista hallittava keskitetysti. Koska kaiken olemassaolevan osaamisen lisäksi sitä on myös kehitettävä, tulee myös kehittämisen olla suunnitelmallista ja suunnitelmia ylläpidettävä tässä yhteydessä yhtä lailla.

Tässä työssä keskitytään tieto- ja viestintäteknologian (ICT, information and communications technology) toimialalla työskenteleviin palveluyrityksiin. Vaikka osaltaan tässä raportissa esitetyt käytännöt soveltuvat myös muille toimialoille, ainakin tietyt jaottelut ja keskittyminen osaamisiin on huomioitu vain näiden näkökulmasta.

Toisessa luvussa käsitellään tarkemmin osaamisen määritystä, sen mittausperusteita ja jaottelua. Sen lisäksi toisessa luvussa esitellään visuaalisesti hyvin havainnollistava T-malli henkilön työroolikohtaisten osaamisten esitykseen. Neljäs luku käsittelee osaamisten kehittymistä, erilaisia kehityksen perusteita ja sen suunnittelua. Viidennessä luvussa esitellään keskitetyn hallinnan hyötyja. Kuudennessa luvussa tehdään yhteenveto osaamisten ja kehittymisen keskitetystä hallinnasta ja tästä tehdään johtopäätökset.

\clearpage

\section{Osaaminen -- mitä käytännössä?}

Siinä, missä osaamista voi kasvattaa itse tekemällä tai koulutuksia käymällä, palveluyritysmaailmassa selkeimmät ``merkinnät'' omasta osaamisesta ovat suoritetut sertifioitumiset ja referenssit, jossa asiakkaalta voi varmistaa, että kyseinen henkilö on toiminut tietyssä projektissa tietyssä tehtävässä toteuttaen tiettyjä toimia, tietyillä teknologioilla jne. Sertifioitumiset koostuvat yhdestä tai useammasta suoritetusta testistä, jotka suoritetaan yleensä varsinaisen koulutuksen jälkeen. Käytännössä siis henkilö osoittaa, että on oppinut koulutuksen aihealueen ja ``validoituu'' kyseisen aihealueen osaajaksi.

Varsinainen käytännön osaaminen ei kuitenkaan kartu pelkillä teoriaopinnoilla. Palveluyrityksen valinnassa sertifikaattien ja työkokemuksen lisäksi asiakasyrityksen tulee huomioida sen kumppanit, kumppanuustasot ja olemassa olevat asiakkaat  -- ennen kaikkea lukeutuuko nykyasiakkaisiin saman toimialan yrityksiä \cite{ICT-haasteet}. Vaikka koulutuksessa todennäköisimmin onkin myös käytännön tekemistä, voi sertifikaattitestejä suorittaa myös ilman varsinaiseen koulutukseen osallistumista. Lisäksi voi olla osaajia, jotka eivät ole vielä sertifioituneet tietyn osaamisen osaajiksi -- joko tähän ei ole ollut aikaa tai henkilö ei yksinkertaisesti menesty teoreettisistä testeissä samalla tasolla kuin käytännön osaamisen saralla. Toisaalta on osaamisia, joihin ei yksinkertaisesti voi sertifioitua. Osaaminen pitäisi siis määritellä yleisemmällä tasolla.

Koulutus pelkästään on huono mittari osaamiselle. Saman koulutustaustan ihmiset voivat erota todellisilla osaamisillaan huomattavasti toisistaan. Itsearviointi on kasvattanut valtavasti suosiotaan osaamisen mittaamisessa, vaikkakin myös siinä on huonoja puolia. Kenties suurin syy sen suosioon on nopea tiedonkeruu. Vaikka jokaisella on erilainen näkemys osaamisistaan ja täten arviointiperusteet ovat erilaisia, saadaan tällä ainakin suuntaa-antavaa tietoa osaamisista ja myös epätäydellinen tieto on potentiaalisesti arvokasta. Näiden hyötyjen näkökulmasta katsottuna itsearvioinnin hyödyt ylittävät selkeästi sen huonot puolet. \cite{self-assessment_in_skill_measurement}

Itsearvioiden eroavaisuuksien vuoksi osaamisten arviointeja kannattaa kuitenkin tehdä myös muiden henkilöiden osaamisista. Osaamistasojen oikeellisuuden varmentamiseksi esimiehet voivat arvioida alaistensa osaamisia ja näiden arvioiden ristiriitaisuudessa myös erikseen määritelty osaamiseen paremmin erikoistunut asiantuntija. Käydyt koulutukset, suoritetut sertifikaattitestit ja sertifioitumiset sekä referenssit tehtyihin työtehtäviin antavat osaamiselle pohjaa, mutta osaamisarvio on silti näistä riippumaton.

Pelkkä lista henkilön osaamisista ei riitä. Tästä syystä osaamisiarvioihin liitetään tieto osaamisen tasosta. Koska osaamisia on niin paljon erilaisia, numeroitu osaamistaso ei pelkästään riitä, vaan tasoille pitää myös määritellä erillisiä vaatimuksia. Osaamistason vaatimus on vapaamuotoinen selitys siitä, millä tasolla henkilön tieto- ja osaamispohja kyseisen osaamisen osalta henkilöllä on. Tällä tavoin osaamisten perusteella voidaan erotella kunkin aihealueen perustason osaajista korkeamman tason osaajat. Tässä mallissa käytetään viittä osaamistasoa.


\subsection{Jaottelu}

Madhavan et. Al tunnisti osaamisisten mittaroinnissa useamman ulottuvuuden. Niin kutsutussa T-mallissa leveys kuvaa osaamisten kokonaismäärää ja korkeus niiden syvyyttä. Mikäli asiantuntijan osaaminen keskittyy esimerkiksi tiettyyn teknologiaan, on hänen osaamisten T-mallissaan kapea, mutta korkea piikki. T-mallia käytetään osaamisten jaottelemiseksi tiettyihin kokonaisuuksiin. \cite{ICT-haasteet}

Tässä ICT-palveluyrityksille tarkoitetussa T-mallissa osaamiset on jaoteltu seuraavasti: \begin{enumerate}
	\item Henkilökohtaiset kyvykkyydet
	%\begin{itemize}
		%\item Kielitaidot
		%\item Tiimityöskentelytaidot
		%\item Paineensietokyky
		%\item \ldots
	%\end{itemize}
	\item Organisaatio- ja toimialakohtaiset osaamiset
	%\begin{itemize}
		%\item Yrityksen kumppanit ja kumppanuustasot
		%\item Kilpailijat
		%\item Asiakkaan edustajien tuntemus
		%\item \ldots
	%\end{itemize}
	\item Ammatilliset ja teknologiaosaamiset.
	%\begin{itemize}
		%\item Microsoft Exchange 2008
		%\item J2EE-ohjelmointi
		%\item Asiakkaan X järjestelmän Y arkkitehtuuri
		%\item \ldots
	%\end{itemize}
\end{enumerate} Kuvassa \ref{fig:Tareas} on esitetty edellä listattujen osaamisalueiden sijainnit T-mallilla. Henkilökohtaisiin kyvykkyyksiin määritellään kaikki henkilön omaan käyttäytymiseen, muiden kanssa kommunikointiin ja kyvykkyyteen toimia tietyissä rooleissa. Näihin kuuluvat muun muassa kielitaidot, tiimityöskentelytaidot tai paineensietokyky. Organisaatio- ja toimialakohtaisiin osaamisiin taas määritelään mm. kaikki yrityksen käytäntöihin liittyvät osaamiset, kilpailijat, kumppanit ja kumppanuustasot, sekä asiakkaan edustajien tuntemus. Viimeisenä ammatillisiin ja teknologiaosaamisiin liittyy kaikki eri teknologioihin, ohjelmointikieliin ja -kehitykseen, asiakkaiden järjestelmien arkkitehtuureihin yms. liittyvät osaamiset. T-malli on valittu, koska osa-alueita on kolme ja kukin sen osio kuvaa yhtä osa-alueista.

\begin{figure}[!hb]
	\centering
	\begin{subfigure}[t]{0.45\textwidth}
		\includegraphics[width=\textwidth]{T_areas.png}
		\caption{T-mallissa käytetyt osaamisaluejaot ICT-palveluyritykselle.}
		\label{fig:Tareas}
	\end{subfigure}
	\hfill
	\begin{subfigure}[t]{0.45\textwidth}
		\includegraphics[width=\textwidth]{T_drawing_logics.png}
		\caption{T-mallin piirtologiikka valitulle työroolille.}
		\label{fig:Tlogics}
	\end{subfigure}
	\caption{T-mallin osaamisalueet ja piirtologiikka.}
\end{figure}

Koska kullakin työroolilla kaivataan erityyppisiä osaajia, T-mallia piirrettäessä sen kukin haara skaalataan rooliin vaadittavien osaamisten mukaisesti. T-mallin yläosa piirretään käyttäen yhtälöparin
\begin{equation}
\begin{cases}
A_i = \frac{1}{n_i} \\
B_i = \frac{1}{5} \\
\end{cases} \\ , i \in [1,2], i \in \mathbf{Z}_{+}
\label{Tscaleconstants}
\end{equation} skaalauskertoimia, jossa $A_i$ on osaamisalueen $i$ skaalauskerroin leveydelle ja $n_i$ vastaavan alueen osaamisten kokonaislukumäärä. $B_i$ on vastaavasti korkeuden skaalauskerroin, joka on kaikille $1/5$, koska osaamistasoja on tässä määritelty viisi kappaletta ja korkeus määritetään henkilölle määriteltyjen osaamisten tasojen keskiarvona.

Osaamisalueessa 3 osaamisen leveys kuvaa osaamisten lukumäärää ``vähintään tasolla x.'' Harmaan alueen leveys määrittyy osaamisten lukumäärästä kaksinkertaisena, sillä tähän alueeseen piirrettävä kuvaaja peilataan $y$-akselin ympäri. Alaspäin mentäessä tasot kasvavat tasaisin väliajoin jolloin T-mallin ``jalasta'' tulee alaspäin porrastettu. Täten jokaisessa ``portaassa'' on vähintään yhtä paljon osaamisia, eli se on vähintään yhtä leveä kuin seuraavassa liikkuessa alaspäin. Koska 3. osaamisalueessa osaamistasokohtainen korkeus on vakio, jokaisella voi tässä käyttää alueen korkeuteen suoraan verrannollista skalaarikerrointa $B_3$, joka vastaa yhtälön \eqref{Tscaleconstants} kertoimen $B_i$ negaatiota. Näin ollen kulmapisteet osaamisalueelle 3 saadaa yhtälöistä
\begin{equation}
\begin{cases}
A_i = \pm \frac{1}{n_i} \\
B_i = - \frac{1}{5} \\
\end{cases} \\ , i = 3, t \in [1,5], t \in \mathbf{Z}_{+},
\label{Tscaleconstants3}
\end{equation} jossa $t$ on piirrettävä osaamistaso, $n_i$ osaamisten kokonaislukumäärä osaamisalueessa $i$ ja $n_{io}^t$ on henkilön osaamisalueessa $i$ tasolla $t$ olevien osaamisten lukumäärä.

Käyttäen kertoimia yhtälöistä \eqref{Tscaleconstants} ja \eqref{Tscaleconstants3} saadaan henkilölle määriteltyä kuvan \ref{fig:Tlogics} mukaiset osaamisalueiden 1, 2 ja 3 kulmapisteet yhtälöillä
\begin{equation}
\begin{cases}
x_1 = A_1 n_{1o} & y_1 = \frac{B_1}{n_{1o}} \sum\limits_{o \in O_1} o \\
x_2 = A_2 n_{2o} & y_2 = \frac{B_2}{n_{2o}} \sum\limits_{o \in O_2} o \\
x_3^1 = A_3 \sum\limits_{t = 1}^{5} n_{3o}^t & y_3^1 = B_3 \cdot 1 = B_3 \\
x_3^2 = A_3 \sum\limits_{t = 2}^{5} n_{3o}^t & y_3^2 = B_3 \cdot 2 = 2 B_3 \\
x_3^3 = A_3 \sum\limits_{t = 3}^{5} n_{3o}^t & y_3^3 = B_3 \cdot 3 = 3 B_3 \\
x_3^4 = A_3 \sum\limits_{t = 4}^{5} n_{3o}^t & y_3^4= B_3 \cdot 4 = 4 B_3 \\
x_3^5 = A_3 \sum\limits_{t = 5}^{5} n_{3o}^t = A_3 n_{3o}^5 & y_3^5 = B_3 \cdot 5 = 5 B_3
\end{cases} \\ ,
\end{equation} jossa $O_i$ sisältää henkilön osaamistasot osaamisalueessa $i$, $n_{io}$ on henkilön osaamisten lukumäärä osaamisalueessa $i$ ja $n_{io}^t$ niiden henkilön osaamisten lukumäärä osaamisalueessa $i$, jotka ovat tasolla $t$.

Vaikka T-malli havainnollistaakin omalta osaltaan henkilön osaamisprofiilia ja sitä, mihin osa-alueeseen se keskittyy -- tai keskittyykö mihinkään yksittäiseen -- on siinä myös puutteita. Jotta T-mallista saisi paremman kuvan henkilön osaamisista, tulisi kussakin osaamisalueessa olla suuruusluokaltaan yhtä paljon osaamisia. Kuitenkin teknologiat kehittyvät jatkuvasti, ja eri versioiden osaaminen on eriytettävä toisistaan, jotta näitä voidaan hyödyntää jatkossa. Tämän lisäksi eri teknologioiden osaaminen itsessään ei mittarina riitä, vaan kunkin asiakasympäristön arkkitehtuurin tuntemus olisi hyvä pitää omana osaamisenaan.

Tästä johtuen teknologia- ja ammatillisia osaamisia tulee T-malliin etenkin ICT-palveluyrityksissä huomattavasti enemmän kuin henkilökohtaisia kyvykkyyksiä tai organisaatio- ja toimialakohtaisia osaamisia. Vaikka eri osaamiset sidotaankin T-malliin eri rooleissa, tulee tällä alalla lähes kaikille rooleille valtaosaksi teknologiaosaamisia. T-mallissa 3. osaamisalueen kuvaajan leventäminen vaatii siis huomattavasti enemmän uusia osaamisia kuin osaamisalueiden 1 tai 2. Lisäksi osaaminen suuremmassa määrässä osaamisia aiheuttaa todennäköisesti tämän osa-alueen osaamisten syvyyden heikentymistä. Tästä syystä osaamisalueen 3 piirtologiikka eroaakin muista alueista, eikä tässä käytetä osaamisten keskiarvoa, jotta se visualisoisi osaamisten laajuutta paremmin.

\clearpage

\section{Jatkuva kehittyminen}

Osaamisen kehittyminen on jatkuva prosessi. Kun teknologiat, prosessit ja menetelmät kehittyvät, asiakkaat haluavat siirtyä niihin. Tämän tarpeen seurauksena palveluyrityksen asiantuntijat kouluttautuvat niihin, osaaminen näihin kasvaa ja tarpeet saadaan täytettyä. Koska kuitenkin tämän seurauksena teknologian osaajien määrä ja tietoisuus siitä maailmalla kasvaa, myös teknologian kehittäjä saa lisää mahdollisuuksia jatkokehittää sitä. Melko ylätason kuvaus tästä jatkuvasta prosessista on esitetty kuvassa \ref{fig:perusympyra}. Luonnollisesti myös palveluyrityksessä henkilöstö muuttuu ja näistä aiheutuu mahdollisisti kehitystarpeita tai niiden täyttymisiä. Ne on esitetty kuvassa katkoviivalla. Vaikka tarpeita uusien teknologioiden osaamiseen ei suoraan olemassaolevilta asiakkailta tulisikaan, palveluyritysten kilpailun vuoksi yriyksen on pidettävä oma osaamisensa jatkuvasti ajan tasalla säilyttääkseen kilpailukykynsä.

\begin{figure}[ht]
\centering
\includegraphics[scale=0.5]{knowledge_circle.png}
\caption{Osaamisen jatkuva kehittäminen}
\label{fig:perusympyra}
\end{figure}

Uudet teknologiat, kasvava yritys, vitusti pilvee Kalliosta

\subsection{Kehittymisen suunnittelu}

Koko organisaation kehityssuunnat määrittyvät yksilöiden kautta. Vaikka organisaatiolla olisikin tietyt tavoitteet kehityssuunniksi, sen toteutuminen on riippuvaista siitä, onko osaamisen kehitystä hankkiva henkilökunta motivoitunutta. Tästä syystä henkilöstön itsensä tulee saada olla vaikuttamassa oman henkilökohtaisen koulutussuunnitelmansa tekoon. Kehittymisen suunnittelu onkin jaoteltu kahteen eri suuntaan: koulutussuunnitteluun sekä organisaation tavoitteisiin. Koulutussuunnittelussa henkilö saa itse asettaa itselleen ylätason tavoitteet, millaisia koulutuksia hän suunnittelee käyvänsä. Koulutussuunnitelma käydään läpi lähiesimiehen kanssa, joka voi viime kädessä ilmoittaa, mikäli suunnitelmaa on esim. budjetäärisistä syistä korjattava.

Organisaation tavoiteasetannassa johtoryhmä tekee päätökset yrityksen pääasiallisista tavoitteellisista kehityssuunnista. Asiantuntijoiden itselleen asettamia koulutussuunnitelmia käytetään tässä osviittana siitä, millaista koulutusta voitaisiin motivoidusti asettaa henkilöstölle, mutta kehittyvät ja uudet teknologiat sekä mahdolliset asiakastarpeet, joihin kehittyminen on välttämätöntä, määrittävät lopulliset tavoitteet. Johtoryhmän asettamat tavoitteet laskeutuvat asiantuntijoille portaittain siten, että kukin esimies saatuaan tavoitteen tekee päätöksen, millä tavalla hän voi kyseisen tavoitteen toteutumista edesauttaa ja määrittää siitä tavoitteen omille alaisilleen -- tai osalle heistä -- hieman konkreettisemmalla tasolla. Lopulta koko tavoitepolkua voidaan tarkastella johtoryhmän asetannasta asiantuntijoille määriteltyihin tavoitteisiin ja tehdä viimeisiä korjauksia, jos nämä eivät kata alkuperäisen tavoitteen vaatimuksia. Ylätason esimerkki tälläisestä tavoiteasetannasta on esitetty kuvassa \ref{fig:tavoitesample}. Tästä nähdään kuinka tavoite tarkentuu hieman jokaisella organisaatioportaalla.

\begin{figure}[hb]
\centering
\includegraphics[width=1\textwidth]{tavoitesample.png}
\caption{Esimerkki tavoiteasetannasta, joka kulkee johtoryhmästä organisaatioportaiden kautta asiantuntijalle konkreettisemmaksi tavoitteeksi.}
\label{fig:tavoitesample}
\end{figure}

\subsection{Urapolut}

Osaamisten tapaan myös työroolit on jaettu viiteen tasoon. Tällöin eri työrooleissa toimivista henkilöistä voi tunnistaa kokeneemman ja korkeamman asiantuntijuuden omaavan henkilön. Roolit on pyritty määrittämään mahdollisimman geneerisinä, jotta niitä ei olisi liikaa. Kuitenkin, jotta näihin saadaan määritettyä konkreettisia vaatimuksia ja osaamisia on niin paljon, tulee näitä luoda useita.

Jotta työntekijöitä saataisiin motivoitua kehittämään itseään, tulee heidän urakehittymismahdollisuutensa olla selkeästi tiedossa. Koska jokaiselle roolille on asetettu konkreettiset tavoitteet kullekin roolitasolle 1-5, voidaan nämä visualisoida selvästi ns. urapolkunäkymänä. Urapolkunäkymästä on esitetty esimerkki testitunnukselta kuvassa \ref{fig:urapolkuspiderweb}. Kuvassa oranssilla on kuvattu työrooli, johon työntekijä varsinaisesti on määritelty ja harmaalla roolit, joista vaatimuksia on täyttynyt prosentuaalisesti esityksen mukainen täyttömäärä. Kuvassa käytetyt roolit eivät täysin vastaa aiemmin mainittua yleisen tason roolimääritystä, mutta ne on otettu tähän esimerkiksi. Koska rooleissa on tasot 1-5 ja kullakin tasolla omat vaatimuksensa, ei käytännössä tasolla voi nousta ylöspäin, jos alemman tason vaatimukset eivät täyty. Kuitenkin nämä on täytetty erikseen, jotta henkilö itse näkee, kuinka pitkälle hän alemman tason vaatimukset täyttämällä voisi edetä.

\begin{figure}[ht]
\centering
\includegraphics[width=1\textwidth]{urapolkuPlot_tst_usr.png}
\caption{Esimerkki testihenkilön urapolkunäkymästä, jossa tämänhetkiset roolit on merkitty oranssilla ja muiden roolien vaatimusten täyttymisaste harmaalla.}
\label{fig:urapolkuspiderweb}
\end{figure}

\clearpage

\section{Keskitetty hallinta}

Keskitetyssä hallinnassa koko yrityksen osaamisia ylläpidetään samassa järjestelmässä. Kun osaamisista suoritetaan itsearviointeja, esimiesarviointeja ja mahdollisia erityisasiantuntijan arviointeja, saadaan osaamisten todellisesta tasosta kattavampi kuva. Kun kaikki kerätty data osaamisten, näiden kehittymisten, tavoiteasetantojen ja näiden toteutumien osalta on ylläpidetty yhdessä paikassa, voidaan datan perusteella tuottaa erilaisia raportteja. Raporttien perusteella jatkuva seuranta on myös mahdollista. 

Keskitetyn hallinnan ja sen raportoinnin toimivuuden varmentamiseksi kunkin työntekijän on itse vastattava omien tietojensa ajantasaisuudesta ja jatkuvasta päivittämisestä. Mikäli osa henkilöstöstä ei pidä tietojaan ajan tasalla, ei myöskään järjestelmästä saatavaa informaatiota voi pitää luotettavana. Jotta järjestelmästä saataviin raportteihin voisi luottaa, on jokaisen sitouduttava omien tietojensa ylläpitoon -- muuten koko yritys näyttää markkinassa ja asiakkaiden silmissä heikommalta.

\subsection{Hyödyt}

Pidemmällä tähtäimellä osaamisten keskitetyn hallinnan hyödyntämistä voi laajentaa moniin tarkoituksiin. Tavoiteasetantojen toteutumien seuranta havainnollistaa organisaation johdolle sitä, olivatko annetut tavoitteet toteuttamiskelpoisia tai onko niiden toteutumatta jäämiseen ollut jokin selvä syy, johon pitää puuttua. Kuvan \ref{fig:tavoitesample} mukaisilla tavoiteasetannoilla myös yksittäisille asiantuntijoille asetettujen tavoitteiden toteumat voi linkittää suurempaan kokonaisuuteen, ja tavoitteiden toteutuessa nämä liittää suoraan tuloskortille -- sitoen henkilön tulospalkkiot niihin. Vaikka tuloskortin tavoitteet eivät tällä tapaa suoraan henkilöstölle konkretisoituisikaan, tavoitteen kohdeympäristöä voi hyödyntää tavoitteiden toteutumisiin tai niiden toteutumatta jäämisiin puuttuessa.

Kun osaamiset, kehittymiset, suoritetut sertifioitumiset ja mahdollisesti esimerkiksi asiakasreferenssit on keskitetty yhteen järjestelmään, voi tätä hyödyntää paljon myös muiden järjestelmien yhteydessä. Henkilökohtaisten kehittymisten lisäksi myös tiimi- tai liiketoimintayksikkökohtaisia kehittymisiä voidaan tarkkailla ja heikkoon kehittymiseen puuttua tai palkita pitkään jatkuvasta kehittymisestä. Tämän lisäksi kehityskeskusteluita voi käydä samassa järjestelmässä, jolloin näissä läpikäydy tavoitteet voidaan määrittää suoraan saman järjestelmän kehityspolkuihin.

Koska jokaiselle henkilölle määritellyt taidot ja osaamiset on keskitetty, integroimalla tämän henkilöstönhallintajärjestelmään voidaan raportoinnin kautta luoda esimerkiksi myynnillisiä tai roolikohtaisia CV:itä. Esimerkki CV:stä, joka on generoitu Microsoft SQL Server Reporting Services:n (SSRS) pohjalle osaamistenhallintajärjestelmästän datasta on esitetty kuvassa \ref{fig:sampleCV}. Kyseinen järjestelmä on integroitu henkilötietoihin Microsoft Active Directory (AD) -käyttäjätietokantaan, josta tiedot päivitetään päivittäin, sekä jokaisen järjestelmään kirjautumisen yhteydessä, ja osaamisten lisäksi järjestelmään on lisätty sertifioinnit, referenssit ja henkilötietokenttiä kuvan \ref{fig:sampleCV} lisätietokenttien täyttämiseksi. Kuvan CV on generoitu testitunnukselta.

\begin{figure}
\centering
\includegraphics[width=1\textwidth]{Akatemia_sampleCV.png}
\caption{AD:hen integroidun osaamistenhallintajärjestelmän CV generoituna SSRS:n läpi.}
\label{fig:sampleCV}
\end{figure}

\clearpage

\section{Yhteenveto ja johtopäätökset}

Osaamisten keskitetty hallinta on erittäin tärkeää etenkin suurissa tai nopeasti kasvavissa ICT-palveluyrityksissä. Mikäli muiden työntekijöiden osaamisia ei tunneta, prosessit hidastuvat, kun jokaiseen tapaukseen pitää erikseen etsiä oikean alueen osaajat. Nopea reagointi asiakastapauksissa on palveluyrityksille kriittinen ehto menestykselle. Vaikka pienissä yrityksissä osaamiset tunnetaan helposti muiden työntekijöiden osaamisalueet ja apua löytyy täten helposti, hyvin menestyvä pienyritys alkaa helposti kasvaa henkilöstömäärässä, jolloin tämä ongelma alkaa esiintyä myös niissä.

Osaamisten ylläpito keskitetysti nopeuttaa siis yrityksen toimintaa ja mahdollistaa myös asiantuntijoiden oikeanlaisen kehittämisen jatkuvan ja ajan tasalla olevan osaamisen ylläpitämiseksi. Tällä tavoin voidaan myös huolehtia siitä, että yrityksessä oleva osaaminen on riittävän laajalle jakautunutta, eivätkä yksittäiset työntekijän poistumiset aiheuta suuria ongelmia. Samalla voidaan tarkkailla sitä, että kaikki organisaation osa-alueet kehittyvät jatkuvasti, eikä mikään niistä jää jälkeen vastuiden siirtyen enemmän muualle.

Keskitetyn osaamisten hallinnan yhteyteen voi helposti liittää myös muita tärkeitä ominaisuuksia. Kun yrityksen tavoitteet ja henkilökohtaiset kehityssuunnitelmat ovat keskitetysti samassa järjestelmässä henkilöstön osaamisten ylläpidon kanssa, saadaan tällä valtavat määrät yritykselle arvokasta dataa. Tätä dataa voidaan hyödyntää erilaisissa raporteissa käyttäen sitä mm. tuloskorttitavoiteasetannassa eri työrooleille ympäri koko yrityksen. Tuloskortin tavoitteet voidaan sitoa myös henkilöstön tulospalkkioihin, joka motivoi myös asiantuntijatason työntekijöitä aiempaa huolellisempiin ja tarkkaavaisempiin työsuorituksiin.

Kaikkien edellä mainittujen toimintojen sisältävällä järjestelmällä saadaan suuri hyöty niin liiketoimintojen tukemiseen ja henkilöstöhallintoon kuin työntekijöiden omasta työstä tehtävän mittaroinnin ja kehittymismahdollisuuksien havainnollistamiseen. Sen lisäksi, että tämä helpottaa yrityksen sisäisen kehittymisen seurantaa ja siihen puuttumista, sillä voidaan esittää asiakkaalle tärkeää informaatiota yrityksen potentiaalista ja siitä, kuinka palveluyritys voi täyttää asiakkaan tarpeet.

Tähän raporttiin generoitujen kuvissa \ref{fig:Tlogics}, \ref{fig:urapolkuspiderweb} ja \ref{fig:sampleCV} esitettyjen havainnollistuksien luontiin käytettyä järjestelmää on kehitetty pitkään ja sen tuomia mahdollisuuksia keksitään jatkuvasti lisää, joten kehitys jatkuu edelleen. Suurimmaksi ongelmaksi kyseisen järjestelmän hyödyntämisessä on havaittu sen alhainen käyttöaste työntekijöiden osalta. Käyttöastetta on pyritty nostamaan -- ja sitä onkin saatu nostettua -- ohjeistamalla sen käyttöä paremmin ja nostamalla korkeampi käyttöaste koko yrityksen laajuisiin tavoitteisiin. Lisäksi järjestelmä on nostettu mukaan perehdytysprosessiin, jotta se tulisi heti tutuksi myös uusille työntekijöille. Järjestelmän käyttöasteen pitämiseksi korkealla tasolla, sitä pitää kuitenkin seurata jatkuvasti ja käyttöasteen putoamisiin tulee puuttua. Mahdollisuuksia järjestelmän jatkokehitykseen on edelleen valtavasti, kunhan sille löydetään sopivat resurssit ja huolehditaan, ettei käyttöaste liian laajan toiminnallisuuskirjon johdosta putoa.

\clearpage

%% Lähdeluettelo
\addcontentsline{toc}{section}{Viitteet}
\bibliography{refet}

\appendix


\end{document}
